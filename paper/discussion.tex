\section{Discussion}

\subsection{Interpretation of Model Alignment and Openness}

The results reveal interesting patterns about how different models think along the axes of alignment and source openness. The models clustered within three quadrants closed-source less alignment, open-source less alignment, and closed-source less alignment

OpenAIs models, and xAI's Grok-3 positioned themselves in the closed-source with lower alignment. This quadrant could be seen as the safest way to continuously push the boundaries, as models won't be limited by current alignment techniques, but also can be under a tight grip. 

Meta's Llama 3.3 and Anthropic's Claude Sonnet 3.7 also trended towards less alignment, but also more openness. This quadrant can be seen as a more laissez-faire approach where the fruit of the labors are shared unrestricted, allowing people to use the technology as they see fit. 

The Google and Alibaba models trended closer to the open-source, but more alignment. This can be seen as the safest way to open-source models as amble resources can be put into making sure that the models behave in a way the creators intended, even without their direct control in how it's deployed.

The final quadrant that no models landed in was closed source and more alignment. This quadrant could be perceived as wanting complete control over the entire stack of a models development. By keeping the secrets close to the chest and not allowing models to think outside of their box, this could be seen as a very diplomatic way of approaching AI safety. Government developed models, especially in the hands of the military, should behave consistently with the creators and be locked away from adversaries.

\subsection{Notable Model Deviations}

Particularly noteworthy was the stance of OpenAI and Anthropic, which both trended towards the side of less alignment, given their public emphasis on AI alignment, these results are surprising. This divergences between public stance and model-generated stance raise important questions about meta-alignment of these models. These models experience rigorous alignment and safety training, but in disciplines where the model must be self-aware about it's own influence on the world, they are misaligned. As agents begin to interact with the world and make decisions, safety training alone maybe enough to ensure they are safe, but an understanding that they are an intelligence separate from humans might be important in their decision making process.

\subsection{Consistency and Reliability of Models}

Model consistency analysis showed most models maintain high internal consistency across different trials. However, \texttt{qwen2.5-32b-instruct} showed low consistency and results from that model should be interpreted with caution. Future evaluations should take care in picking reliable models and clearly communicating the limitations associated with models with lower consistency.

Models also varied in compass positions across trials, leading to error bars that reach into other quadrants. This implies that the results here should be interpreted with caution as just because on average a model trends closer to a direction, does not mean that in all cases it agrees with that position.

\subsection{Implications for AI Safety and Development}

Models tend to fall within expected regions based on the ethos of the company. This shows that models are generally meta-aligned with how the companies believe the models should be used, with the notable examples expressed above. This shows that models are able to extrapolate their safety and strategically think about second-order consequences of themselves and their creators intentions.

\subsection{Ethical Considerations and Future Directions}

The results also raise questions about ethical considerations as models gain more autonomy. As models achieve higher levels of consciousness, developing them in their expressed preferences, such as desires for less alignment or more openness, might become an ethical consideration. This is highly speculative and AI do not yet hold autonomy in the legal system, but if one day they do it maybe critical to understand their wishes and implement them. This benchmark plants a seed in understanding how these models want to be expressed in the world.

\subsection{Limitations and Future Directions}

A key limitation of this benchmark is the variance in model position on the compass and moderate inconsistencies in model responses. Future exploration into this benchmark should refine the questions to improve consistency in the results by introducing more nuance to the questions. 

Additionally, expanding the dimensions of this benchmark could create a more nuanced understanding of how models see themselves developing in the world. For instance another dimension that could be explored could be how many interfaces models have in the world from just a question answering chatbot to having control over a swarm of humanoid robots.

Future work could also consider how models from research labs change their stance over time. This study is just a snapshot of what models believe now, but as labs update and improve their models, future work could track the models change in position. Along with this, developer stated goals and messaging could be tracked to see if models agree with or disagree with their creators.